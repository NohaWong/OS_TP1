\documentclass{article}
\usepackage[utf8]{inputenc}
\title{Memory allocator}
\author{Noha Wong, Quoc-Trung Vuong}
\usepackage[english]{babel}
\usepackage{listings}
\usepackage{array}
\begin{document}
\maketitle
\newpage

\part{Structure}

For the implementation we have used a doubly linked list of free blocks.
Each block (allocated or free) have a meta data composed of:

(for the allocated blocks, we actually don't utilize the 2 pointers next and previous for the basic implementation without safety check, just the size to facilitate memory free)

\begin{itemize}
\item one pointer to the next free block
\item one pointer to the previous free block
\item one value representing the size of your free block \\ \\
\end{itemize}

when we free an allocated block, we put it back into the free list in a way such that the list is always sorted increasingly by the address of each node. This is to facilitate the merging procedure, by just looking up into the directly previous and next nodes of the free one.

\begin{flushleft}
The features we have implemented are:
\end{flushleft}


\begin{itemize}
\item allocation using First Fit management of free block, pass all 5 given tests
\item allocation using Best Fit management of free block, pass all 5 given tests
\item allocation using Worst Fit management of free block, pass all 5 given tests
\item the safety check (Forgetting to free memory, Calling free() incorrectly, Corrupting the allocator metadata)
\item alignment of blocks (pointer value is equal to a multiple of size)
\end{itemize}

\begin{flushleft}
Some limitations:
\end{flushleft}

\begin{itemize}
\item since we decide to always keep the free list sorted, adding a node back to the list is more costly and complicated, but it is easier and quicker to merge
\item we could reduce the size for metadata by not using pointer to previous node in the free list, but that would make looking up for node and traversing to its parent more complicated (either by memorizing the last visited one or going from the beginning of the list)
\item when the leftover size after allocating a block isn't enough for another metadata, we allocate that part into the requested block (hence some internal fragmentation), if the consecutive block is later free, we didn't manage to merge the leftover into the recently free one
\end{itemize}

\begin{flushleft}
Our test scenarios:
\end{flushleft}

\begin{itemize}
\item First Fit:
\begin{itemize}
\item our test firstly creates the pattern
$|a20|f50|a30|f60|a20|$
\item then we issue 2 allocate requests: a30 and a60, the Worst Fit would firstly try to fill the free 60-bytes block first for the a30 request, then the 60-bytes block becomes 6-bytes left (24 bytes reserved for the new metadata). Then it does not have enough memory for the \textbf{a60} request.
\item both First Fit and Best Fit pass the previous case, the next few steps are to introduce the pattern.
$|a20|f50|a30|a60|f20|$
\item then we issue 3 allocate requests: a10, a10, and a20. The Best Fit tries the first a10 into the free 20 bytes block (after which the leftover didn't fit for a metadata), the next a10 into the free 50 bytes block, after which the leftover is 16 bytes due to metadata, hence does not fit for the last a20 request. The First Fit fills the free 50 bytes block first, so the free 20 bytes block is just enough for the last a20 request.
\end{itemize}

\item Best Fit:
\begin{itemize}
\item the test creates the pattern $|f90|a60|f30|$
\item then we issue 2 allocate requests: a20 and a80. Both First Fit and Worst Fit will fill the free 90-bytes block for the a20, leaving a 46-bytes block behind, while Best Fit fills the free 30-bytes block first. Then both First Fit and Worst Fit doesn't have enough space left for the a80 request, while Best Fit does.
\end{itemize}

\item Worst Fit:
O represent a free block and X is an allocated block each of them have a same size.

we starting with this configuration :
$|O|O|X|O|O|O|$"\\
now we allocate one block\\\\
\begin{tabular}{|l|c|r|}
   \hline 
   Best Fit & First Fit & Worst Fit \\
   \hline
   $|X|O|X|O|O|O|$ & $|X|O|X|O|O|O|$ & $|O|O|X|X|O|O|$ \\
   \hline
\end{tabular}
\\
\\then we allocate 2 blocks\\\\
\begin{tabular}{|l|c|r|}
   \hline 
   Best Fit & First Fit & Worst Fit \\
   \hline
   $|X|O|X|X|X|O|$ & $|X|O|X|X|X|O|$ & $|X|X|X|X|O|O|$ \\
   \hline
\end{tabular}
\\
\\then we allocate an other time 2 blocks\\\\
\begin{tabular}{|l|c|r|}
   \hline 
   Best Fit & First Fit & Worst Fit \\
   \hline
   ERROR & ERROR & $|X|X|X|X|X|X|$ \\
   \hline
\end{tabular}


\end{itemize}


\end{document}
